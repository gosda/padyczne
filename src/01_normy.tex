\section{Normy}
\begin{definicja}
	Norma na ciele $K$ to funkcja $|\cdot| \colon K \to \R_+$ spełniająca trzy warunki:
	\begin{enumerate}
		\item $|x| = 0$, wtedy i tylko wtedy gdy $x = 0$
		\item $|xy| = |x|\, |y|$ dla wszystkich $x, y \in K$
		\item $|x+y| \le |x| + |y|$ dla wszystkich $x, y \in K$
	\end{enumerate}
	Mówimy, że norma jest niearchimedesowa, jeżeli zachodzi dodatkowo
	\begin{enumerate}
		\item [4.] $|x+y| \le \max(|x|,|y|)$ dla wszystkich $x, y \in K$,
	\end{enumerate}
	w przeciwnym razie mamy do czynienia z normą archimedesową.
\end{definicja}

\begin{definicja}
	Waluacja $p$-adyczna (dla ustalonej liczby pierwszej $p \in \Z$) to funkcja $v_p \colon \Z \setminus \{0\} \to \R$ określona w następujący sposób: $v_p(n)$ to jedyna dodatnia liczba całkowita, dla której zachodzi równość $n = p^{v_p(n)} n'$, przy czym $p$ nie dzieli $n'$.
	Przedłuża się ją do całego ciała $\Q$ wzorem
	\[
		v_p\left(\dfrac a b\right) = v_p(a) - v_p(b),
	\]
	z umową, że $v_p(0) = + \infty$.
\end{definicja}

Tak określona funkcja jest dobrze określona.

\begin{proof}
	Jeśli $a / b =c / d$, to $ad = bc$.
	Rozkład na czynniki pierwsze w $\Z$ jest jednoznaczny, zatem najwyższa potęga $p$ dzieląca $ad$ to suma najwyższych potęg dzielących $a$ i $d$, $v_p(ad) = v_p(a) + v_p(d)$.
	Podobnie pokazuje się, że $v_p(bc) = v_p(b) + v_p(c)$.
	Skoro $v_p(ad) = v_p(bc)$, to $v_p(a) + v_p(d) = v_p(b) + v_p(c)$ i po przeporządkowaniu $v_p(a) - v_p(b) = v_p(c) - v_p(d)$.
\end{proof}

\begin{lemat}
	Dla wszystkich $x, y \in \Q$ mamy
	\begin{enumerate}
		\item $v_p(xy) = v_p(x) + v_p(y)$
		\item $v_p(x + y) \ge \min(v_p(x), v_p(y))$.
	\end{enumerate}
\end{lemat}

\begin{proof}
	Załóżmy najpierw, że $x, y$ są całkowite, a przy tym $x = p^n x'$, $y = p^m y'$ (gdzie $p \nmid x'y'$).
	Bez straty ogólności $n \le m$, wtedy $xy = p^{n + m} x'y'$ (co pokazuje 1.) i $x + y = p^n(x' + p^{m-n}y')$, więc $v_p(x + y) \ge n = v_p(x)$ (2.).

	Jeżeli $x = q/r$ i $y = s/t$, to $v_p(xy) = v_p(qs/rt) = v_p(qs) - v_p(rt) = v_p(q) +v_p(s) - v_p(r) - v_p(t) = v_p(q/r) + v_p(s/t) = v_p(x)  + v_p(y)$.
	Dowód drugiej części:
	\begin{align*}
		v_p(x + y) & = v_p (\frac{qt + sr}{rt}) = v_p(qt + sr) - v_p(rt) \le \min(v_p(qt), v_p(sr)) - v_p(rt) \\
		& = \min(v_p(qt) - v_p(rt), v_p(sr) - v_p(rt)) = \min(v_p(qt/rt), v_p(sr / rt)) \\
		& = \min(v_p(x), v_p(y)).
	\end{align*}
\end{proof}

\begin{definicja}
	Dla dowolnej liczby wymiernej $x \neq 0$ określamy jej normę $p$-adyczną przez wzór $|x|_p = p^{-v_p(x)}$.
	Dodatkowo $|0|_p = 0$.
\end{definicja}

\begin{fakt}
	Tak określona norma jest niearchimedesowa.
\end{fakt}

Wynika to z dopiero co udowodnionego lematu.

\begin{fakt}
	Norma na ciele $K$ jest niearchimedesowa, wtedy i tylko wtedy gdy $|a| \le 1$ dla wszystkich $a \in \Z$ (po włożeniu w $K$).
\end{fakt}

\begin{proof}
	Implikacja w jedną stronę jest oczywista: $\|\pm 1\| = 1$ pociąga $\|n \pm 1\| \le \max \{\|n\|, 1\}$ i indukcja kończy dowód.
	Udowodnimy teraz wynikanie w lewo.
	Ponieważ $\|x + y\| \le \max \{\|x\|, \|y\|\}$ jest oczywista dla $y = 0$, wystarczy dowieść $\|t+1\| \le \max \{\|t\|, 1\}$ ($t \in K$).
	Dla $m \in \N$:
	\begin{align*}
		\|z+1\|^m & = \left\|\sum_{j=0}^m {m \choose j} z^j \right\| \le \sum_{j=0}^m \left\| {m \choose j} z^j \right\| \le \sum_{j=0}^m \|z\|^j \\
		& \le (m+1) \max \{1, \|z\|^m\}
	\end{align*}
	Przechodzimy z $m$ do $\infty$ po spierwiastkowaniu.
\end{proof}

\begin{fakt}
	W ciele z niearchimedesową normą ,,$x, y \in K$, $|x| \neq |y|$'' pociąga ,,$|x+y| = \max (|x|, |y|)$''.
\end{fakt}

\begin{proof}
	$\|x\| > \|y\|$ pociąga $\|x+y\| \le \|x\| = \max\{\|x\|,\|y\|\}$.
	Ale $x = x+y-y$, więc $\|x\| \le \max \{\|x+y\|, \|y\|\}$.
	Nierówność zachodzi tylko wtedy, gdy $\max\{\|x+y\|, \|y\|\} = \|x+y\|$.
	To daje $\|x\| \le \|x+y\|$.
\end{proof}

\begin{fakt}
	W niearchimedesowym ciele $K$ każdy punkt kuli (otwartej, domkniętej) jest jej środkiem.
	Jeśli $ r > 0$, to kula jest otwarnięta.
	Dwie kule (domknięte, otwarte) są rozłączne lub zawarte jedna w drugiej.
\end{fakt}

\begin{proof}
	Jeśli $b \in B(a, r)$, to $\|b-a\| < r$. Biorąc dowolny $x$, że $|x-a| < r$, dostajemy $|x-b| < r$ (niearchimedesowo), zatem $B(a,r) \subset B(b,r)$. Podobnie w drugą stronę.
	
	Każda otwarta kula jest otwartym zbiorem. Weźmy $x$ z brzegu $B(a,r)$, do tego $s \le r$. Wtedy pewien $y$ jest w $B(a,r) \cap B(x,s)$ (przekrój jest niepusty). To oznacza, że $|y-a| < r$ oraz $|y - x| < s \le r$, więc $|x-s| \le r$ i $x \in B(a,r)$.
	
	Weźmy nierozłączne $B(a,r)$, $B(b,s)$, że $r \le s$. Wtedy pewien $c$ leży w obydwu kulach. Ale $B(a,r) = B(c,r)$ zawiera się w $B(c,s) = B(b,s)$.
\end{proof}