\section{Twierdzenie Ostrowskiego}
\begin{lemat}
	Wartości bezwzględne $\|\cdot\|_i$ na $K$ są równoważne wtedy i tylko wtedy, gdy $\|x\|_1 < 1 \Leftrightarrow \|x\|_2<1$ (inaczej: dla pewnej $\alpha > 0$ i każdego $x$ zachodzi $\|x\|_1 = \|x\|_2^\alpha$).
	Tutaj $i = 1, 2$.
\end{lemat}

\begin{proof}
	Dowód polegał będzie na pokazaniu ciągu implikacji.
	\begin{itemize}
		\item [$3 \Rightarrow 1$] $\|x-a\|_1 < r$ wtedy i tylko wtedy, gdy $\|x-a\|_2 < r^{1/\alpha}$; ,,otwarte kule są nadal otwarte''. 
		\item [$1 \Rightarrow 2$] Dla równoważnych wartości bezwzględnych mamy jedną zbieżność; $\lim_n x^n = 0$ jest równoważne $\|x\| < 1$.
		\item [$2 \Rightarrow 3$] 	Wybierzmy $x_0 \in K$ różne od $0$, że $|x_0|_1 < 1$.
	Warunek nr 2 mówi, że $|x_0|_2$ też jest mniejsze od jeden, czyli możemy wybrać $\alpha > 0$ takie, żeby $|x_0|_1 = |x_0|_2^\alpha$.
	\end{itemize}

	Wybierzmy jeszcze jeden $x \in K \setminus \{0\}$.
	Jeśli $|x|_1 = |x_0|_1$, to $|x|_2 = |x_1|_2$ (gdyby tak nie było, to normy ilorazów byłyby zepsute).
	Podobnie dla $|x|_1 = 1$.

	Bez straty ogólności zakładamy, że $1 > |x|_1 \neq |x_0|_1$.
	Znów istnieje $\beta > 0$, że $|x|_1 = |x|_2^\beta$, ale czy $\alpha = \beta$?
	Niech $n$, $m$ będą naturalne.
	Wtedy $|x|_1^n < |x_0|_1^m \iff |x|_2^n < |x_0|_2^m$.
	Wzięcie logarytmów daje (po drobnych przekształceniach)
	\[
		\frac nm < \frac{\log |x_0|_1}{\log |x|_1} \iff \frac n m < \frac{\log |x_0|_2}{\log |x|_2}.
	\]

	Oznacza to, że ułamki po prawych stronach są równe.
	Po podłożeniu $|x_0|_1 = |x_0|_2^\alpha$ okaże się, że rzeczywiście $\alpha = \beta$.
\end{proof}

\begin{twierdzenie}[Ostrowski, 1916]
	Na $\Q$ wartość bezwzględna musi być równoważna z jedną z wartości bezwzględnych $\|\cdot\|_p$, gdzie $p$ jest l. pierwszą lub $p = \infty$ (lub dyskretną).
\end{twierdzenie}

\begin{proof}
	Niech $|\cdot|$ będzie nietrywialną normą na $\Q$.
	Pierwszy przypadek: archimedesowa (odpowiada jej $|\cdot|_\infty$).
	Weźmy więc najmniejsze dodatnie całkowite $n_0$, że $|n_0| > 1$.
	Wtedy $|n_0| = n_0^\alpha$ dla pewnej $\alpha > 0$.
	Wystarczy uzasadnić, dlaczego $|x| = |x|_\infty^\alpha$ dla każdej $x \in \Q$, a właściwie tylko dla $x \in \Z_{>0}$ (bo norma jest multiplikatywna).
	Dowolną liczbę $n$ można zapisać w systemie o podstawie $n_0$: $n = a_0 + a_1 n_0 + \dots + a_kn_0^k$, gdzie $a_k \neq 0$ i $0 \le a_i \le n_0-1$.
	\begin{align*}
	|n| & = \left|\sum_{i=0}^k a_in_0^i\right| \le \sum_{i=0}^k \left|a_i\right| n_0^{i \alpha} \le n_0^{k \alpha} \sum_{i = 0}^k n_0^{-i \alpha} \\ & \le n_0^{k \alpha} \sum_{i = 0}^\infty n_0^{-i \alpha} = n_0^{k \alpha} \frac{n_0^\alpha}{n_0^\alpha - 1} = C n_0^{k \alpha}
	\end{align*}

	Pokazaliśmy $|n| \le Cn_0^{k \alpha} \le C n^\alpha$ dla każdego $n$, a więc w szczególności dla liczb postaci $n^N$ (bowiem $C$ nie zależy od $n$): $|n| \le C^{1/n}n^\alpha$.
	Przejdźmy z $N$ do nieskończoności, dostajemy $C^{1/n} \to 1$ i $|n| \le n^\alpha$.
	Teraz trzeba pokazać nierówność w drugą stronę.
	Skorzystamy jeszcze raz z rozwinięcia.
	Skoro $n_0^{k+1} > n \ge n_0^k$, to zachodzi
	\[
		n_0^{(k+1)\alpha} = |n_0^{k+1}| = |n+n_0^{k+1} - n| \le |n| + |n_0^{k+1} - n|,
	\]
	a stąd wnioskujemy, że 
	\[
		|n| \ge n_0^{(k+1)\alpha} - |n_0^{k+1}-n| \ge n_0^{(k+1)\alpha} - (n_0^{k+1}-n)^\alpha.
	\]

	Skorzystaliśmy tutaj z nierówności udowodnionej wyżej.
	Wiemy, że $n \ge n_0^k$, więc prawdą jest, że
	\begin{align*}
		|n| & \ge n_0^{(k+1)\alpha} - (n_0^{k+1} - n_0^k)^\alpha \\
		& = n_0^{(k+1) \alpha} [1 - (1 - \textstyle \frac{1}{n_0})^\alpha]  = C' n_0^{(k+1)\alpha} > C' n^\alpha.
	\end{align*}

	Od $n$ nie zależy $C' = 1 - (1-1/n_0)^\alpha$, jest dodatnia i przez analogię do poprzedniej sytuacji możemy pokazać $|n| \ge n^\alpha$.
	Wnioskujemy stąd, że $|n| = n^\alpha$ i $|\cdot|$ jest równoważna ze zwykłą wartością bezwzględną.

	Załóżmy, że $|\cdot|$ jest niearchimedesowa.
	Wtedy $\|n\| \le 1$ dla całkowitych $n$.
	Ponieważ $|\cdot|$ jest nietrywialna, musi istnieć najmniejsza l. całkowita $n_0$, że $\|n_0\| < 1$.
	Zacznijmy od tego, że $n_0$ musi być l. pierwszą: gdyby zachodziło $n_0 = a \cdot b$ dla $1 < a,b < n_0$, to $|a| = |b| = 1$ i $|n_0| < 1$ (z minimalności $n_0$) prowadziłoby do sprzeczności.
	Chcemy pokazać, że $|\cdot|$ jest równoważna z normą $p$-adyczną, gdzie $p := n_0$.
	W następnym kroku uzasadnimy, że jeżeli $n \in \Z$ nie jest podzielna przez $p$, to $|n| = 1$.
	Dzieląc $n$ przez $p$ z resztą dostajemy $n = rp + s$ dla $0 < s < p$.
	Z minimalności $p$ wynika $|s| = 1$, zaś z $|r| \le 1$ ($|\cdot|$ jest niearchimedesowa) i $|p| < 1$: $|rp| < 1$.
	,,Wszystkie trójkąty są równoramienne'', więc $|n| = 1$.
	Wystarczy więc tylko zauważyć, że dla $n \in \Z$ zapisanej jako $n = p^v n'$ z $p \nmid n'$ zachodzi $|n| = |p|^v |n'| = |p|^v = c^{-v}$, gdzie $c = |p|^{-1} > 1$, co kończy dowód.
\end{proof}

\begin{fakt}[,,adelic product'']
	Jeżeli $x \in \Q^\times$, to $\prod_{p \le \infty} |x|_p = 1$.
\end{fakt}