\section{Uzupełnianie}

\begin{lemat}
	Ciało $\Q$ z nietrywialną normą nie jest zupełne.
\end{lemat}

\begin{proof}
	Dzięki twierdzeniu Ostrowskiego wystarczy sprawdzić $p$-adyczne normy.
	Niech $p \neq 2$ będzie pierwsza, zaś $a \in \Z$ taka, że nie jest kwadratem, nie dzieli się przez $p$ i równanie $x^2 = a$ ma rozwiązanie w $\Z/p\Z$.
	Konstruujemy ciąg Cauchy'ego bez granicy: $x_0$ jest dowolnym rozwiązaniem równania, $x_n$ ma być równe $x_{n-1}$ modulo $p^n$ oraz $x_n^2 = a$ (modulo $p^{n+1}$).
	Jest Cauchy'ego ($|x_{n+1} - x_n| = |\lambda p^{n+1}| \le p^{-n+1} \to 0$) i nie ma granicy (kandydatem na nią jest pierwiastek z $a$, gdyż prosty rachunek pokazuje $|x_n^2 - a| = |\mu p^{n+1}| \le p^{-n+1} \to 0$).
	Gdy $p = 2$, to zastępujemy pierwiastek  sześciennym.
\end{proof}

Zbiór ciągów Cauchy'ego oznaczmy przez $C$.
Można na nim zadać strukturę pierścienia (przemienego i z jedynką) przez punktowe dodawanie oraz mnożenie.
Wprowadzamy ideał $N$, do którego należą ciągi zbieżne do zera.

\begin{lemat}
	Zbiór $N$ jest ideałem maksymalnym $C$.
\end{lemat}

\begin{proof}
Ustalmy ciąg $(x_n) \in C \setminus N$ oraz ideał $I = \langle (x_n), N \rangle$.
Od pewnego miejsca $x_n$ nie jest zerem, zatem $y_n = 1/x_n$ od tego miejsca i $y_n = 0$ ma sens.
Ciąg $y_n$ jest Cauchy'ego:
\[
	|y_{n+1} - y_n| = \frac{|x_{n+1} - x_n|}{|x_nx_{n+1}|} \le \frac{|x_{n+1}-x_n|}{c^2} \to 0.
\]
Ale $(1) - (x_n)(y_n) \in N$, to kończy dowód ($I = C$).
\end{proof}

\begin{definicja}
	Ciało liczb $p$-adycznych to $\Q_p := C  / N$.
\end{definicja}

\begin{lemat}
	Ciąg $|x_n|_p$ jest stacjonarny, gdy $(x_n) \in C \setminus N$.
\end{lemat}

\begin{proof}
	Można znaleźć takie liczby $c, N_1$, że $n \ge N_1$ pociąga $|x_n| \ge c > 0$.
	Z drugiej strony istnieje taka $N_2$, że $n, m \ge N_2$ pociąga $|x_n - x_m| < c$.
	Połóżmy więc $N = \max\{N_1, N_2\}$.
	Wtedy $n, m \ge N$ pociąga $|x_n - x_m| < \max\{|x_n|, |x_m|\}$, a to oznacza, że $|x_n| = |x_m|$.
\end{proof}

Dzięki temu następująca definicja nie jest bez sensu:

\begin{definicja}
	Gdy $(x_n) \in C$ reprezentuje $\lambda \in \Q_p$, przyjmujemy $|\lambda|_p := \lim_{n \to \infty} |x_n|_p$.
\end{definicja}

\begin{lemat}
	Obraz $\Q \hookrightarrow \Q_p$ po włożeniu jest gęsty.
\end{lemat}

\begin{proof}
	Chcemy pokazać, że każda otwarta kula wokół $\lambda \in \Q_p$ kroi się z obrazem $\Q$, czyli zawiera ,,stały ciąg''.
	Ustalmy kulę $B(\lambda, \varepsilon)$, ciąg Cauchy'ego $(x_n)$ dla $\lambda$ i $\varepsilon' < \varepsilon$.
	Dzięki temu, że ciąg jest Cauchy'ego, możemy znaleźć dla niego indeks $N$, że $n, m \ge N$ pociąga $|x_n - x_m| < \varepsilon'$.
	Rozpatrzmy stały ciąg $(y)$ dla $y = x_N$.
	Wtedy $|\lambda - (y)| < \varepsilon$, gdyż $\lambda - (y)$ odpowiada ciąg $(x_n-y)$.
	Ale $|x_n - x_N| < \varepsilon'$ i w granicy
	\[
		\lim_{n \to \infty}|x_n - y| \le \varepsilon' < \varepsilon. \qedhere
	\]
\end{proof}

\begin{fakt}
	Ciało $\Q_p$ jest zupełne.
\end{fakt}

\begin{proof} Dowód w czterech krokach:
	\begin{enumerate}
		\item Niech $\lambda_k$ będzie ciągiem Cauchy'ego elementów $\Q_p$. \item Skoro obraz $\Q$ w $\Q_p$ jest gęsty, to można znaleźć liczby wymierne $l_k$, że $\lim_{n \to \infty} |\lambda_n - (l_n)| = 0$: granica w $\Q_p$!
		\item Wybrane wcześniej liczby wymierne $l_n$ same tworzą ciąg Cauchy'ego w $\Q$; dążą do $\lambda$ w $\Q_p$.
		\item Zachodzi $\lim_{n \to \infty} \lambda_n = \lambda$. \qedhere
	\end{enumerate}
\end{proof}