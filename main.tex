\documentclass[a4paper, fleqn]{extreport}
\usepackage{amsmath, amsfonts, amsthm, amssymb}

\usepackage{lmodern}
\usepackage{euler}
\usepackage{geometry}
\usepackage{microtype}

\newcommand{\N}{\mathbb{N}}
\newcommand{\Z}{\mathbb{Z}}
\newcommand{\Q}{\mathbb{Q}}
\newcommand{\R}{\mathbb{R}}
\newcommand{\C}{\mathbb{C}}

\newcounter{dummy}
\numberwithin{dummy}{section}
%\theoremstyle{definition}
\newtheorem{przyklad}[dummy]{Przyk\l{}ad}
\newtheorem{fakt}[dummy]{Fakt}
\newtheorem{lemat}[dummy]{Lemat}
\newtheorem{definicja}[dummy]{Definicja}
\newtheorem{wniosek}[dummy]{Wniosek}
\newtheorem{twierdzenie}[dummy]{Twierdzenie}

\usepackage[polish]{babel}
\usepackage[utf8]{inputenc}
\usepackage[T1]{fontenc}
\selectlanguage{polish}

\author{R. S.}
\title{Liczby $p$-adyczne}

\begin{document}
\maketitle
\tableofcontents

\chapter{Nieuporządkowane}

\section{Normy}
\begin{definicja}
	Norma na ciele $K$ to funkcja $|\cdot| \colon K \to \R_+$ spełniająca trzy warunki:
	\begin{enumerate}
		\item $|x| = 0$, wtedy i tylko wtedy gdy $x = 0$
		\item $|xy| = |x|\, |y|$ dla wszystkich $x, y \in K$
		\item $|x+y| \le |x| + |y|$ dla wszystkich $x, y \in K$
	\end{enumerate}
	Mówimy, że norma jest niearchimedesowa, jeżeli zachodzi dodatkowo
	\begin{enumerate}
		\item [4.] $|x+y| \le \max(|x|,|y|)$ dla wszystkich $x, y \in K$,
	\end{enumerate}
	w przeciwnym razie mamy do czynienia z normą archimedesową.
\end{definicja}

\begin{definicja}
	Waluacja $p$-adyczna (dla ustalonej liczby pierwszej $p \in \Z$) to funkcja $v_p \colon \Z \setminus \{0\} \to \R$ określona w następujący sposób: $v_p(n)$ to jedyna dodatnia liczba całkowita, dla której zachodzi równość $n = p^{v_p(n)} n'$, przy czym $p$ nie dzieli $n'$.
	Przedłuża się ją do całego ciała $\Q$ wzorem
	\[
		v_p\left(\dfrac a b\right) = v_p(a) - v_p(b),
	\]
	z umową, że $v_p(0) = + \infty$.
\end{definicja}

Tak określona funkcja jest dobrze określona (udowodnić).

\begin{lemat}
	Dla wszystkich $x$ oraz $y \in \Q$ mamy
	\begin{enumerate}
		\item $v_p(xy) = v_p(x) + v_p(y)$
		\item $v_p(x + y) \ge \min(v_p(x), v_p(y))$.
	\end{enumerate}
\end{lemat}

\begin{definicja}
	Dla dowolnej liczby wymiernej $x \neq 0$ określamy jej normę $p$-adyczną przez wzór $|x|_p = p^{-v_p(x)}$.
	Dodatkowo $|0|_p = 0$.
\end{definicja}

\begin{fakt}
	Tak określona norma jest niearchimedesowa.
\end{fakt}

\begin{fakt}
	Norma na ciele $K$ jest niearchimedesowa, wtedy i tylko wtedy gdy $|a| \le 1$ dla wszystkich $a \in \Z$ (po włożeniu w $K$).
\end{fakt}

\begin{fakt}
	W ciele z niearchimedesową normą ,,$x, y \in K$, $|x| \neq |y|$'' pociąga ,,$|x+y| = \max (|x|, |y|)$''.
\end{fakt}

\begin{proof}
	$\|x\| > \|y\|$ pociąga $\|x+y\| \le \|x\| = \max\{\|x\|,\|y\|\}$.
	Ale $x = x+y-y$, więc $\|x\| \le \max \{\|x+y\|, \|y\|\}$.
	Nierówność zachodzi tylko wtedy, gdy $\max\{\|x+y\|, \|y\|\} = \|x+y\|$.
	To daje $\|x\| \le \|x+y\|$.
\end{proof}

\begin{fakt}
	W niearchimedesowym ciele $K$ każdy punkt kuli (otwartej, domkniętej) jest jej środkiem.
	Jeśli $ r > 0$, to kula jest otwarnięta.
	Dwie kule (domknięte, otwarte) są rozłączne lub zawarte jedna w drugiej.
\end{fakt}

\begin{proof}
	\begin{enumerate}
		\item Jeśli $b \in B(a, r)$, to $\|b-a\| < r$. Biorąc dowolny $x$, że $|x-a| < r$, dostajemy $|x-b| < r$ (niearchimedesowo), zatem $B(a,r) \subset B(b,r)$. Podobnie w drugą stronę.
		\item Każda otwarta kula jest otwartym zbiorem. Weźmy $x$ z brzegu $B(a,r)$, do tego $s \le r$. Wtedy pewien $y$ jest w $B(a,r) \cap B(x,s)$ (przekrój jest niepusty). To oznacza, że $|y-a| < r$ oraz $|y - x| < s \le r$, więc $|x-s| \le r$ i $x \in B(a,r)$.
		\item Weźmy nierozłączne $B(a,r)$, $B(b,s)$, że $r \le s$. Wtedy pewien $c$ leży w obydwu kulach. Ale $B(a,r) = B(c,r)$ zawiera się w $B(c,s) = B(b,s)$. \qedhere
	\end{enumerate}
\end{proof}

\section{Twierdzenie Ostrowskiego}
\begin{lemat}
	Wartości bezwzględne $\|\cdot\|_i$ na $K$ są równoważne wtedy i tylko wtedy, gdy $\|x\|_1 < 1 \Leftrightarrow \|x\|_2<1$ (inaczej: dla pewnej $\alpha > 0$ i każdego $x$ zachodzi $\|x\|_1 = \|x\|_2^\alpha$).
	Tutaj $i = 1, 2$.
\end{lemat}

\begin{proof}
	Dowód polegał będzie na pokazaniu ciągu implikacji.
	\begin{itemize}
		\item [$3 \Rightarrow 1$] $\|x-a\|_1 < r$ wtedy i tylko wtedy, gdy $\|x-a\|_2 < r^{1/\alpha}$; ,,otwarte kule są nadal otwarte''. 
		\item [$1 \Rightarrow 2$] Dla równoważnych wartości bezwzględnych mamy jedną zbieżność; $\lim_n x^n = 0$ jest równoważne $\|x\| < 1$.
		\item [$2 \Rightarrow 3$] 	Wybierzmy $x_0 \in K$ różne od $0$, że $|x_0|_1 < 1$.
	Warunek nr 2 mówi, że $|x_0|_2$ też jest mniejsze od jeden, czyli możemy wybrać $\alpha > 0$ takie, żeby $|x_0|_1 = |x_0|_2^\alpha$.
	\end{itemize}

	Wybierzmy jeszcze jeden $x \in K \setminus \{0\}$.
	Jeśli $|x|_1 = |x_0|_1$, to $|x|_2 = |x_1|_2$ (gdyby tak nie było, to normy ilorazów byłyby zepsute).
	Podobnie dla $|x|_1 = 1$.

	Bez straty ogólności zakładamy, że $1 > |x|_1 \neq |x_0|_1$.
	Znów istnieje $\beta > 0$, że $|x|_1 = |x|_2^\beta$, ale czy $\alpha = \beta$?
	Niech $n$, $m$ będą naturalne.
	Wtedy $|x|_1^n < |x_0|_1^m \iff |x|_2^n < |x_0|_2^m$.
	Wzięcie logarytmów daje (po drobnych przekształceniach)
	\[
		\frac nm < \frac{\log |x_0|_1}{\log |x|_1} \iff \frac n m < \frac{\log |x_0|_2}{\log |x|_2}.
	\]

	Oznacza to, że ułamki po prawych stronach są równe.
	Po podłożeniu $|x_0|_1 = |x_0|_2^\alpha$ okaże się, że rzeczywiście $\alpha = \beta$.
\end{proof}

\begin{twierdzenie}[Ostrowski, 1916]
	Na $\Q$ wartość bezwzględna musi być równoważna z jedną z wartości bezwzględnych $\|\cdot\|_p$, gdzie $p$ jest l. pierwszą lub $p = \infty$ (lub dyskretną).
\end{twierdzenie}

\begin{proof}
	Niech $|\cdot|$ będzie nietrywialną normą na $\Q$.
	Pierwszy przypadek: archimedesowa (odpowiada jej $|\cdot|_\infty$).
	Weźmy więc najmniejsze dodatnie całkowite $n_0$, że $|n_0| > 1$.
	Wtedy $|n_0| = n_0^\alpha$ dla pewnej $\alpha > 0$.
	Wystarczy uzasadnić, dlaczego $|x| = |x|_\infty^\alpha$ dla każdej $x \in \Q$, a właściwie tylko dla $x \in \Z_{>0}$ (bo norma jest multiplikatywna).
	Dowolną liczbę $n$ można zapisać w systemie o podstawie $n_0$: $n = a_0 + a_1 n_0 + \dots + a_kn_0^k$, gdzie $a_k \neq 0$ i $0 \le a_i \le n_0-1$.
	\begin{align*}
	|n| & = \left|\sum_{i=0}^k a_in_0^i\right| \le \sum_{i=0}^k \left|a_i\right| n_0^{i \alpha} \le n_0^{k \alpha} \sum_{i = 0}^k n_0^{-i \alpha} \\ & \le n_0^{k \alpha} \sum_{i = 0}^\infty n_0^{-i \alpha} = n_0^{k \alpha} \frac{n_0^\alpha}{n_0^\alpha - 1} = C n_0^{k \alpha}
	\end{align*}

	Pokazaliśmy $|n| \le Cn_0^{k \alpha} \le C n^\alpha$ dla każdego $n$, a więc w szczególności dla liczb postaci $n^N$ (bowiem $C$ nie zależy od $n$): $|n| \le C^{1/n}n^\alpha$.
	Przejdźmy z $N$ do nieskończoności, dostajemy $C^{1/n} \to 1$ i $|n| \le n^\alpha$.
	Teraz trzeba pokazać nierówność w drugą stronę.
	Skorzystamy jeszcze raz z rozwinięcia.
	Skoro $n_0^{k+1} > n \ge n_0^k$, to zachodzi
	\[
		n_0^{(k+1)\alpha} = |n_0^{k+1}| = |n+n_0^{k+1} - n| \le |n| + |n_0^{k+1} - n|,
	\]
	a stąd wnioskujemy, że 
	\[
		|n| \ge n_0^{(k+1)\alpha} - |n_0^{k+1}-n| \ge n_0^{(k+1)\alpha} - (n_0^{k+1}-n)^\alpha.
	\]

	Skorzystaliśmy tutaj z nierówności udowodnionej wyżej.
	Wiemy, że $n \ge n_0^k$, więc prawdą jest, że
	\begin{align*}
		|n| & \ge n_0^{(k+1)\alpha} - (n_0^{k+1} - n_0^k)^\alpha \\
		& = n_0^{(k+1) \alpha} [1 - (1 - \textstyle \frac{1}{n_0})^\alpha]  = C' n_0^{(k+1)\alpha} > C' n^\alpha.
	\end{align*}

	Od $n$ nie zależy $C' = 1 - (1-1/n_0)^\alpha$, jest dodatnia i przez analogię do poprzedniej sytuacji możemy pokazać $|n| \ge n^\alpha$.
	Wnioskujemy stąd, że $|n| = n^\alpha$ i $|\cdot|$ jest równoważna ze zwykłą wartością bezwzględną.

	Załóżmy, że $|\cdot|$ jest niearchimedesowa.
	Wtedy $\|n\| \le 1$ dla całkowitych $n$.
	Ponieważ $|\cdot|$ jest nietrywialna, musi istnieć najmniejsza l. całkowita $n_0$, że $\|n_0\| < 1$.
	Zacznijmy od tego, że $n_0$ musi być l. pierwszą: gdyby zachodziło $n_0 = a \cdot b$ dla $1 < a,b < n_0$, to $|a| = |b| = 1$ i $|n_0| < 1$ (z minimalności $n_0$) prowadziłoby do sprzeczności.
	Chcemy pokazać, że $|\cdot|$ jest równoważna z normą $p$-adyczną, gdzie $p := n_0$.
	W następnym kroku uzasadnimy, że jeżeli $n \in \Z$ nie jest podzielna przez $p$, to $|n| = 1$.
	Dzieląc $n$ przez $p$ z resztą dostajemy $n = rp + s$ dla $0 < s < p$.
	Z minimalności $p$ wynika $|s| = 1$, zaś z $|r| \le 1$ ($|\cdot|$ jest niearchimedesowa) i $|p| < 1$: $|rp| < 1$.
	,,Wszystkie trójkąty są równoramienne'', więc $|n| = 1$.
	Wystarczy więc tylko zauważyć, że dla $n \in \Z$ zapisanej jako $n = p^v n'$ z $p \nmid n'$ zachodzi $|n| = |p|^v |n'| = |p|^v = c^{-v}$, gdzie $c = |p|^{-1} > 1$, co kończy dowód.
\end{proof}

\begin{fakt}[,,adelic product'']
	Jeżeli $x \in \Q^\times$, to $\prod_{p \le \infty} |x|_p = 1$.
\end{fakt}

\section{Uzupełnianie}

\begin{lemat}
	Ciało $\Q$ z nietrywialną normą nie jest zupełne.
\end{lemat}

\begin{proof}
	Dzięki twierdzeniu Ostrowskiego wystarczy sprawdzić $p$-adyczne normy.
	Niech $p \neq 2$ będzie pierwsza, zaś $a \in \Z$ taka, że nie jest kwadratem, nie dzieli się przez $p$ i równanie $x^2 = a$ ma rozwiązanie w $\Z/p\Z$.
	Konstruujemy ciąg Cauchy'ego bez granicy: $x_0$ jest dowolnym rozwiązaniem równania, $x_n$ ma być równe $x_{n-1}$ modulo $p^n$ oraz $x_n^2 = a$ (modulo $p^{n+1}$).
	Jest Cauchy'ego ($|x_{n+1} - x_n| = |\lambda p^{n+1}| \le p^{-n+1} \to 0$) i nie ma granicy (kandydatem na nią jest pierwiastek z $a$, gdyż prosty rachunek pokazuje $|x_n^2 - a| = |\mu p^{n+1}| \le p^{-n+1} \to 0$).
	Gdy $p = 2$, to zastępujemy pierwiastek  sześciennym.
\end{proof}

Zbiór ciągów Cauchy'ego oznaczmy przez $C$.
Można na nim zadać strukturę pierścienia (przemienego i z jedynką) przez punktowe dodawanie oraz mnożenie.
Wprowadzamy ideał $N$, do którego należą ciągi zbieżne do zera.

\begin{lemat}
	Zbiór $N$ jest ideałem maksymalnym $C$.
\end{lemat}

\begin{proof}
Ustalmy ciąg $(x_n) \in C \setminus N$ oraz ideał $I = \langle (x_n), N \rangle$.
Od pewnego miejsca $x_n$ nie jest zerem, zatem $y_n = 1/x_n$ od tego miejsca i $y_n = 0$ ma sens.
Ciąg $y_n$ jest Cauchy'ego:
\[
	|y_{n+1} - y_n| = \frac{|x_{n+1} - x_n|}{|x_nx_{n+1}|} \le \frac{|x_{n+1}-x_n|}{c^2} \to 0.
\]
Ale $(1) - (x_n)(y_n) \in N$, to kończy dowód ($I = C$).
\end{proof}

\begin{definicja}
	Ciało liczb $p$-adycznych to $\Q_p := C  / N$.
\end{definicja}

\begin{lemat}
	Ciąg $|x_n|_p$ jest stacjonarny, gdy $(x_n) \in C \setminus N$.
\end{lemat}

\begin{proof}
	Można znaleźć takie liczby $c, N_1$, że $n \ge N_1$ pociąga $|x_n| \ge c > 0$.
	Z drugiej strony istnieje taka $N_2$, że $n, m \ge N_2$ pociąga $|x_n - x_m| < c$.
	Połóżmy więc $N = \max\{N_1, N_2\}$.
	Wtedy $n, m \ge N$ pociąga $|x_n - x_m| < \max\{|x_n|, |x_m|\}$, a to oznacza, że $|x_n| = |x_m|$.
\end{proof}

Dzięki temu następująca definicja nie jest bez sensu:

\begin{definicja}
	Gdy $(x_n) \in C$ reprezentuje $\lambda \in \Q_p$, przyjmujemy $|\lambda|_p := \lim_{n \to \infty} |x_n|_p$.
\end{definicja}

\begin{lemat}
	Obraz $\Q \hookrightarrow \Q_p$ po włożeniu jest gęsty.
\end{lemat}

\begin{proof}
	Chcemy pokazać, że każda otwarta kula wokół $\lambda \in \Q_p$ kroi się z obrazem $\Q$, czyli zawiera ,,stały ciąg''.
	Ustalmy kulę $B(\lambda, \varepsilon)$, ciąg Cauchy'ego $(x_n)$ dla $\lambda$ i $\varepsilon' < \varepsilon$.
	Dzięki temu, że ciąg jest Cauchy'ego, możemy znaleźć dla niego indeks $N$, że $n, m \ge N$ pociąga $|x_n - x_m| < \varepsilon'$.
	Rozpatrzmy stały ciąg $(y)$ dla $y = x_N$.
	Wtedy $|\lambda - (y)| < \varepsilon$, gdyż $\lambda - (y)$ odpowiada ciąg $(x_n-y)$.
	Ale $|x_n - x_N| < \varepsilon'$ i w granicy
	\[
		\lim_{n \to \infty}|x_n - y| \le \varepsilon' < \varepsilon. \qedhere
	\]
\end{proof}

\begin{fakt}
	Ciało $\Q_p$ jest zupełne.
\end{fakt}

\begin{proof} Dowód w czterech krokach:
	\begin{enumerate}
		\item Niech $\lambda_k$ będzie ciągiem Cauchy'ego elementów $\Q_p$. \item Skoro obraz $\Q$ w $\Q_p$ jest gęsty, to można znaleźć liczby wymierne $l_k$, że $\lim_{n \to \infty} |\lambda_n - (l_n)| = 0$: granica w $\Q_p$!
		\item Wybrane wcześniej liczby wymierne $l_n$ same tworzą ciąg Cauchy'ego w $\Q$; dążą do $\lambda$ w $\Q_p$.
		\item Zachodzi $\lim_{n \to \infty} \lambda_n = \lambda$. \qedhere
	\end{enumerate}
\end{proof}

\chapter{Lemat Hensela}

\begin{twierdzenie}[lemat Hensela]
	Niech $\mathfrak K$ będzie ciałem zupełnym względem wartości bezwzględnej $|\cdot|$ i niech $f(X) \in \mathfrak O[X]$.
	Załóżmy, że $a_0 \in \mathfrak O$ spełnia nierówność $|f(a_0)| < |f'(a_0)|^2$, gdzie $f'(X)$ jest (formalną) pochodną.
	Wtedy istnieje $a \in \mathfrak O$, taki że $f(a) = 0$.
\end{twierdzenie}

\begin{proof}
	Niech wielomiany $f_j(X)$ (dla $j = 1, 2, \ldots$) będą zdefiniowane przez tożsamość
	\[
		f(X + Y) = f(X) + \sum_{j \ge 1} f_j(X) Y^j
	\]
	dla niezależnych niewiadomych $X$, $Y$.
	Wtedy $f_1(X) = f'(X)$.
	Ponieważ $|f(a_0)| < |f'(a_0)|^2$, istnieje $b_0 \in \mathfrak O$, takie że $f(a_0) + b_0 f_1(a_0) = 0$.
	Istotnie,
	\[
		|b_0| = \left|\frac{- f(a_0)}{f_1(a_0)} \right| = \frac{|f(a_0)|}{|f_1(a_0)|} < \frac{|f'(a_0)|^2}{|f'(a_0)|} = |f'(a_0)| \le 1.
	\]
	Zgodnie z definicją wielomianów $f_j$ zachodzi relacja
	\[
		|f(a_0 + b_0)| \le \max_{j \ge 2} |f_j(a_0) b_0^j|.
	\]
	Jako że $f_j(X) \in \mathfrak O[X]$ i $a_0 \in \mathfrak O$, mamy $|f_j(a_0)| \le 1$.
	Oznacza to, że 
	\[
		|f(a_0 + b_0)| \le |b_0^2| = \frac{|f(a_0)|^2}{|f'(a_0)|^2} < |f(a_0)|,
	\]
	skorzystaliśmy tu ponownie z nierówności $|f(a_0)| < |f'(a_0)|^2$.

	Podobnie pokazuje się, że
	\[
		|f_1(a_0 + b_0) - f_1(a_0)| \le |b_0| < |f_1(a_0)|,
	\]
	a przez to
	\[
		|f_1(a_0 + b_0)| = |f_1(a_0)|.
	\]
	Kładziemy teraz $a_1 = a_0 + b_0$ i powtarzamy proces.
	Otrzymujemy w ten sposób ciąg $a_n = a_{n- 1} + b_{n- 1}$.
	Dla każdego $n$ prawdziwa jest równość $|f_1(a_n)| = |f_1(a_0)|$, jednocześnie
	\[
		|f(a_{n+1})| \le \frac{|f(a_n)|^2}{|f_1(a_n)|^2} = \frac{|f(a_n)|^2}{|f_1(a_0)|^2} % \le |f(a_n)|^2 \cdot \frac{1}{|f_1(a_0)|^2} \le (\frac{|f(a_{n-1})|^2}{|f_1(a_0)|^2})^2 \cdot \frac{1}{|f_1(a_0)|^2}
	\]
	To uzasadnia zbieżność $f(a_n)$ do zera.
	Co więcej,
	\[
		|a_{n+1} - a_n| = |b_n| = \frac{|f(a_n)|}{|f_1(a_n)|} = \frac{|f(a_n)|}{|f_1(a_0)|} \to 0.
	\]
	Ciąg $\{a_n\}$ jest fundamentalny, z zupełności ciała $\mathfrak K$ wynika istnienie jego granicy oraz $f(a) = 0$.
\end{proof}

\end{document}

analogia do metody Newtona
twierdzenie faktoryzacyjne, Straßmanna
zbieżność jednostajna -> zastosowanie do $\exp_2(2x^2 - 2x)$.
Budowa rozszerzeń.